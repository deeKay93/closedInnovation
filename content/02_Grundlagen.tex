\section{Grundlagen}\label{sec:grundlagen}
\textit{Closed Innovation} ist eine Variante den Wissensaustausch
zur Entwicklung einer \textit{Innovation}
innerhalb eines \textit{Innovationsprozesses} zu organisieren.

In der Literatur finden sich für die beiden letzteren Begriffe verschiedene Beschreibungen.
Im Folgenden werden die in dieser Arbeit verwendeten Definitionen vorgestellt.

\subsection{Innovation}\label{sec:grundlagen-inno}
Bevor die Begriffe \textit{Closed} und \textit{Open Innovation} näher betrachtet werden,
ist es notwendig den Begriff \textit{Innovation} zu definieren.
Es finden sich in der Literatur viele verschiedene Definitionen für diesen Begriff.
\cite[5]{hauschildt2016innovationsmanagement} zeigt eine Auswahl solcher Definitionen und fasst diese
zusammen zu der allgemeinen Aussage:
\begin{quote}
    \enquote{Innovationen sind qualitativ neuartige Produkte oder Verfahren,
    die sich gegenüber einem Vergleichszustand \enquote{merklich} [...] unterscheiden}
    \cite[4]{hauschildt2016innovationsmanagement}
\end{quote}

In \cite[9]{herzog2011} wird ein solches neuartiges Produkt oder Verfahren als \textit{Invention} bezeichnet und erst durch kommerzielle Verwertung zur \textit{Innovation}.
Die vorliegende Arbeit folgt dieser Definition.
Es gilt somit:
\begin{equation*}
    Innovation = Invention + kommerzielle~Verwertung
\end{equation*}

Hierbei ist zu beachten, dass der Begriff sich nicht nur auf vermarktete Produkte bezieht,
sondern auch auf Prozesse, welche innerhalb der Produktion genutzt werden.

\subsection{Innovationsprozess}\label{sec:grundlagen-prozess}
Um Ideen und Innovationen zu entwickeln und zu vermarkten durchlaufen diese zunächst einen Innovationsprozess.
Nach \cite[10\psq]{herzog2011} besteht dieser Prozess aus drei Phasen:
\begin{enumerate}
    \item Im \textit{Frontend of Innovation} werden neue Ideen geschaffen, ausgewählt, sowie auf technologische Machbarkeit und mögliche Marktakzeptanz hin bewertet.
    \item Anschließend werden ausgewählte Ideen realisiert und entwickelt.
    \item Zuletzt wird die Vermarktung der Entwicklungsergebnisse geplant und durchgeführt.
\end{enumerate}
