\section{Fazit}\label{sec:fazit}
Wie in Kapitel \ref{sec:bewertung} gezeigt, überwiegen bei der \textit{Closed Innovation}
die Nachteile klar die Vorteile.
Aufgrund der seinerzeit vorherrschenden Wissenslandschaft und des damaligen Arbeitsmarkts
hatte dieses Paradigma vor allem im 20. Jahrhundert seine Berechtigung.

Doch mittlerweile hat sich dies geändert.
Universitäten sind offener gegenüber praxisorientierter Forschung.
Qualifizierte Arbeitskräfte sind einfacher verfügbar.
Zusätzlich hat die Komplexität der Innovationen so weit zugenommen,
dass das benötigte Wissen durch ein einzelnes Unternehmen nicht mehr erfasst werden kann.

Somit ist \textit{Closed Innovation} vor allem in der IT-Industrie als nicht mehr zeitgemäß anzusehen.