\section{Fazit}\label{sec:fazit}
Vor allem im 20. Jahrhundert ist die \textit{Closed Innovation}
aufgrund der damaligen Wissensumgebung ein logischer Ansatz,
um Innovation in einem Unternehmen zu etablieren.

Mittlerweile hat sich dies geändert.
Universitäten sind offener gegenüber praxisorientierter Forschung.
Qualifizierte Arbeitskräfte sind einfacher verfügbar.
Zusätzlich hat die Komplexität der Innovationen so weit zugenommen,
dass das benötigte Wissen durch ein einzelnes Unternehmen nicht mehr erfasst werden kann.

Gerade in der IT-Industrie ist zudem ein Wandel hin zu Open-Source-Projekten hin zu erkennen.
Beispielsweise gibt es viele verschiedene \linebreak{}\textit{UI-Frameworks}, um das Entwickeln von Webanwendungen zu erleichtern.
Zu den größten zählen das von Google entwickelte \textit{Angular} und Facebooks \textit{React}.
Beide Frameworks werden nicht nur durch das jeweilige Unternehmen,
sondern auch durch eine große Open-Source-Gemeinschaft unterstützt.
Innerhalb von SAP wird jedoch hauptsächlich das eigenentwickelte \textit{SAPUI5} genutzt.
Damit kann das Unternehmen nicht von den Innovationen der anderen Unternehmen profitieren und
durch die Schulung der Mitarbeiter entstehen hohe Kosten.

Wie in Kapitel \ref{sec:bewertung} gezeigt, überwiegen in der heutigen Zeit bei der \textit{Closed Innovation}
die Nachteile klar die Vorteile.
Somit ist \textit{Closed Innovation} vor allem in der IT-Industrie als nicht mehr zeitgemäß anzusehen.

Anstatt jedoch \textit{Closed Innovation} kategorisch zu verwerfen,
ist auch eine Kombination mit \textit{Open Innovation} denkbar \cite{OpenInno32:online}.
Auf diese Weise ist es möglich, die interne Innovationskraft stark aus- und geistiges Eigentum aufzubauen,
während gleichzeitig die Nachteile der \textit{Closed Innovation} kompensiert werden.
