\section{Grundlagen}\label{sec:grundlagen}
In \cite{chesbrough2003} führt Henry W. Chesbrough erstmals die Begriffe \textit{Open} und
\textit{Closed Innovation} ein.
Mit diesen Begriffen bennent er zwei Paradigmen, welche den Innovationsprozess im Hinblick auf die
Interaktion mit der Umwelt des Unternehmens und dem Fluss von Wissen betrachtet.

Diese Seminararbeit behandelt vor allem die \textit{Closed Innovation}.
In den folgenden Unterkapiteln wird zunächst der Begriff \textit{Innovation} im Allgemeinen definiert.
Darufhin werden \textit{Closed Innovation} und grundlegend sein Gegenstück näher betrachtet.


\subsection{Innovation}\label{sec:grundlagen-inno}
Bevor die Begriffe \textit{Closed} und \textit{Open Innovation} näher betrachtet werden,
ist es notwendig den Begriff \textit{Innovation} an sich zu definieren.
Verschiedene Autoren definieren diesen Begriff auf unterschiedliche Arten.
In \cite[5]{hauschildt2016innovationsmanagement} ist eine Auswahl verschiedener Definitionen dargestellt.
Diese werden zusammengefasst zu der allgemeinen Aussage:
\begin{quote}
    \enquote{Innovationen sind qualitativ neuartige Produkte oder Verfahren,
    die sich gegenüber einem Vergleichszustand \enquote{merklich} [...] unterscheiden}
    \cite[4]{hauschildt2016innovationsmanagement}
\end{quote}

In \cite[9]{herzog2011} wird ein solches neuartiges Produkt oder Verfahen als \textit{Invention} bezeichnet und erst durch kommerzielle Verwertung zur \textit{Innovation}.
Die vorliegende Arbeit folgt dieser Definition.
Es gilt somit:
\begin{equation*}
    Innovation = Invention + kommerzielle~Verwertung
\end{equation*}

Hierbei ist zu beachten, dass der Begriff sich nicht nur auf Produkte bezieht, welche vermarktet werden,
sonder auch auf Prozesse, welche innerhalb der Produktion genutzt werden.

\subsection{Innovationsprozess}\label{sec:grundlagen-prozess}
\todo[inline]{Grob den Innovationsprozess beschreiben (Idee -> Ausarbeitung -> Vermarktung)}

\subsection{Closed Innovation}\label{sec:grundlagen-closed}

\paragraph{Ursprung}
Das \textit{Closed Innovation}-Modell ist das vorherschende Innovationsmodell im 20. Jahrhundert.
Es spiegelt die damals vorherschende Wissensumgebung wider.
Unter Wissenschaftlern ist es verpönt ihre Fähigkeiten zu Nutzen um wirtschaftliche Probleme zu lösen.

Um dennoch Innovationen anzutreiben und so den kommerziellen Erfolg zu sichern,
sind Unternhemen gezwungen selbst in \ac{fe} zu investieren.
Da in anderen Unternehmen das notwendige Wissen nicht aufgebaut ist,
müssen Unternehmen innerhalb ihrer \ac{fe}-Organisation das gesamte Spektrum von den Grundlagen bis hin zum fertigen Produkt abdecken.
Voraussetzung hierfür ist die Akquise von talentierten Mitarbeitern.

\todo[inline]{Geschichte etwas ausführlicher}

\paragraph{Prinzipien}
Implizit ergeben sich so die Prinzipien der \textit{Closed Innovation} (nach \cite[19]{herzog2011}):
\begin{itemize}
    \item Ein Unternehmen sollte die talentiertesten Mitarbeiter anstellen.
    \item Um von Innovationen zu profitieren, müssen diese innerhalb des Unternehmens den gesamten Innovationsprozess durchlaufen (Vom Entdecken, über das Entwicklen bis hin zur Vermarktung)
    \item Um ein Produkt als erstes auf den Markt zu bringen, muss eine Entdeckung innerhalb des eingenen Unternehmens entstehen.
    \item Durch die Markteinführung als erstes Unternehmen gewinnt es den Wettbewerb
    \item Investiert ein Unternehmen am meisten in \ac{fe}, so entstehen dort die besten Ideen, was wiederum den Wettbewerb gewinnt.
    \item Damit andere Unternehmen nicht von den eigenen Entdeckungen profitieren, gilt es das geistige Eigentum zu schützen.
\end{itemize}

\paragraph{Modell}
Aus diesen Regeln folgt das in \autoref{fig:closedInnovation} dargestellte Modell.
Der gesamte Innovatoinsprozess spielt sich innerhalb der Grenzen des Unternehmes ab.
Eine Innovation beginnt immer durch eine Idee innerhalb des Unternehmens,
wird durch die firmeneigene \ac{fe}-Abteilung entwickelt
und schließlich auf dem bestehenden Markt des Unternehmens kommerzialisiert.

\begin{figure}[ht!]
    \centering
    \includegraphics[width=1\textwidth]{ClosedInnovation}
    \caption{Closed Innovation Modell (aus \cite[20]{herzog2011})}
    \label{fig:closedInnovation}
\end{figure}

Selbstverständlich können nicht alle Ideen realisiert werden.
Möglicherweise ist die Ursache hierfür fehlende Ressourcen (Personell oder Wissen)
oder es findet sich innerhalb der Unternehmensgrenzen keine vermarktbare Anwendung für eine neue Entdeckung.
Auch werden Ideen gegebenenfalls nicht umgesetzt, da vermutet wird, dass keine Akzeptanz am Markt herrscht.

Abgelehnte Ideen und abgebrochene Projekte werden beispielsweise in Datenbanken gesammelt.
Von dort werden sie wieder aufgegriffen oder bleiben ein ungenutzer Teil des geistigen Eigentums einer Unternehmens.


\subsection{Open Innovation}\label{sec:grundlagen-open}

Neues Modell
Heute muss
Gründe: Wanderung von Fachkräften
Spezielle gebiete nich alleine erforschbar
Hohe Kosten
Uni forscht Praxisnah

Ideen kommen auch von Außen
Nicht umstzbare Ideen werden lizensiert oder outgesourced



\subsection{Kombination}\label{sec:grundlagen-kombi}

Beides lässt sich auch kombinieren