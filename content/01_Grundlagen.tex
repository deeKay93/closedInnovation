\section{Grundlagen}\label{sec:grundlagen}

Open und Closed Innovation grundlegende Konzepte vom Cheeseburger.
Beschreiben Konzepte wie Innovation im Hinblick auf die Umwelt im Unternehmen organisiert werden kann.

Diese Seminarerbeit behandelt vor allem die Closed Innovation,
welche im Folgenden Unterkapitel behandelt wird.
Zusätzlich wird grundlegend die Open innovation zum Vergleich betrachtet.

\subsection{Closed Innovation}\label{sec:grundlagen-closed}

Altes Modell.
Damals muss.
Wegen: Forschung an Uni nicht Praxisnah
Fachkräfte müssen gebunden werden.
Forschung im Eigenen Unternehmen
Ideen nur von Innen
Ideen werden Umgesetzt oder Pausiert.

Keine neuen Märkte

\subsection{Open Innovation}\label{sec:grundlagen-open}

Neues Modell
Heute muss
Gründe: Wanderung von Fachkräften
Spezielle gebiete nich alleine erforschbar
Hohe Kosten
Uni forscht Praxisnah

Ideen kommen auch von Außen
Nicht umstzbare Ideen werden lizensiert oder outgesourced



\subsection{Kombination}\label{sec:grundlagen-kombi}

Beides lässt sich auch kombinieren