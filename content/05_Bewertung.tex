\section{Bewertung der \textit{Closed Innovation}}\label{sec:bewertung}


\subsection{Vorteile}\label{sec:bewertung-vor}
Ein großer Vorteil der \textit{Closed Innovation} ist die vollständige Kontrolle über den Innovaitonsprozess.
Das Unternehmen kann die Entwicklung steuern,
um so die eigenen Ziele zu erreichen.
Durch die weitestgehende Unabhängigkeit von Zulieferern besteht auch keine Abhängigkeit zu deren Innovationskraft.

Als Unternehmen, welches sich in eine komfortablen Markposition befindet
und welches die personellen aber vor allem auch finanziellen Ressourcen besitzt,
kann die interne \ac{fe} ausgebaut und gefördert werden.
Auf diesem Weg können viele Ideen generiert und Innovationen entwicklet werde,
was wiederum die Marktmacht stärk oder -- in stark umkämpften Märkten -- zumindest erhält.

Da die enstehenden Innovationen lediglich als selbst vermarktete Produkte veröffentlicht werden,
kann ich ein innovatives Unternehmen Alleinstellungsmerkmale und somit einen Wettbewerbsvorteil gegenüber anderen,
weniger innovativen, Unternehmen sichern.

\subsection{Nachteile}\label{sec:bewertung-nach}
Um im Unternehmen eine erfolgreiche \ac{fe}-Abteilung zu etablieren,
werden hochqualifizierte Fachkräfte benötigt.
Auch die Konkurrenz wirbt um diese Fachkräfte,
wodurch hohe Kosten für die Akquise und das Halten der Mitarbeiter entstehen.

Zusätzlich ist der Erfolg von \textit{Closed Innovation} davon abhängig,
dass die \enquote{besten Fachkräfte} Teil des eigenen Unternehmens sind.
Praktisch ist dies jedoch nicht möglich.

Nach \cite{stevens19973} werden ca. 3000 Ideen benötogt um am Ende ein erfolgreiches Produkt zu erhalten.
Aus diesen Ideen müssen die relvanten ausgewählt und untersucht werden.
Angepasst auf den dreistufigen Innovaitonsprozess verlassen nur etwa 125 dieser Ideen die erste Stufe
und werden in kleinen Projekten entwickelt.
Im weiteren Verlauf werden von diesen 125 Projekten im Schnitt überschaubare 1,7 Projekte kommerzialisiert.
Es müssen somit viele Ideen gleichzeitig entwicklet werden,
von welchen nur ein geringer Teil später Gewinne einbringt.
Viele der Ideen werden nicht umgesetzt, da deren eigentliches Potential nicht ersichtlich ist.
Insgesamt entstehen somit hohe Kosten und potentielle Gewinne werden nicht erwirtschaftet.

Wie bereits genannt sind Fachkräfte ein großer Faktor für den Erfolg von \textit{Closed Innovation}.
Verlassen diese das Unternehmen, so nehmen sie das gesammelte Wissen mit sich.
Beispielsweise ist es möglich, dass der Mitarbeiter durch einen Konkurrenten abgeworben wird.
Eine andere Ursache für die Fluktuation von Mitarbeitern ist,
dass diese an einem Projekt beteiligt sind,
welches von dem Unternehmen nicht weiter verfolgt werden soll.
Im Gegensatz zum Unternehmen glauben die Mitarbeiter jedoch an den Erfolg ihrer Idee
und entschließen sich diese in einem eigenen oder einem anderen Unternehmen zu realisieren.

Unabhängig von der Art wie der Mitarbeiter das Unternehmen verlässt,
verliert das Unternehmen so einen Teil seines geistigen Eigentums
und kann daraus keine Innovationen und Gewinne mehr generieren.
Schlimmer noch, ein anderes Unternehmen provitiert von den bereits getätigten Investitionen.

In der IT-Branche ist das Wissen auf viele Unternehmen und vor allem zahlreiche Startups verteilt.
Außerdem ist viel Wissen beispielsweise durch Open-Source-Software frei zugänglich.
Wird der Ansatz der \textit{Closed Innovation} streng verfolgt,
so kann auf dieses Wissen jedoch nicht zugegriffen werden.

\subsection{Erwähnenswertes}\label{sec:bewertung-erw}
Vor allem im 20. Jahrhundert ist die \textit{Closed Innovation}
aufgrund der damailigen Wissensumgebung
der einzig logische Ansatz um Innovation in einem Unternehmen zu etablieren.

\textit{Closed} und \textit{Open Innovation} sind zwar an sich gegensätzliche Methoden,
sollten aber nicht unbedingt getrennt voneinander betrachted werden.
Wie in \cite{OpenInno32:online} vorgeschlagen wird,
können die beiden Prinzipen auch kombiniert werden.
Somit kann die interne Innovationskraft stark ausgebaut und geistiges Eigentum aufgebaut werden,
während gleichzeitig die Nachteile der \textit{Closed Innovation} kompensiert werden.
