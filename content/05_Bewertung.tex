\section{Bewertung der \textit{Closed Innovation}}\label{sec:bewertung}


\subsection{Vorteile}\label{sec:bewertung-vor}
Ein großer Vorteil der \textit{Closed Innovation} ist die vollständige Kontrolle über den Innovationsprozess.
Das Unternehmen kann die Entwicklung steuern,
um so die eigenen Ziele zu erreichen.
Durch die weitestgehende Unabhängigkeit von Zulieferern besteht keine Abhängigkeit zu deren Innovationskraft.

Als Unternehmen, welches sich in einer komfortablen Markposition befindet
und welches die personellen aber vor allem auch finanziellen Ressourcen besitzt,
kann die interne \ac{fe} ausgebaut und gefördert werden.
Auf diesem Weg können viele Ideen generiert und Innovationen entwickelt werden.
Dies stärkt wiederum die Markposition oder erhält sie in stark umkämpften Märkten.

Da die entstehenden Innovationen lediglich als selbst vermarktete Produkte veröffentlicht werden,
kann sich ein innovatives Unternehmen Alleinstellungsmerkmale sichern.
Somit entsteht ein Wettbewerbsvorteil gegenüber anderen,
weniger innovativen, Unternehmen.

\subsection{Nachteile}\label{sec:bewertung-nach}
\begin{description}
    \item[Fachkräfte]
        Um im Unternehmen eine erfolgreiche \ac{fe}-Abteilung zu etablieren,
        werden hochqualifizierte Fachkräfte benötigt.
        Auch die Konkurrenz \linebreak{}wirbt um diese Fachkräfte,
        wodurch hohe Kosten für die Akquise und das Halten der Mitarbeiter entstehen.

        Zusätzlich ist der Erfolg von \textit{Closed Innovation} davon abhängig,
        dass die \enquote{besten Fachkräfte} Teil des eigenen Unternehmens sind.
        Praktisch ist dies jedoch nicht möglich.

    \item[Verlust von geistigem Eigentum]
        Verlassen Mitarbeiter das Unternehmen, so nehmen sie das gesammelte Wissen mit sich.
        Beispielsweise ist es möglich, dass der Mitarbeiter durch einen Konkurrenten abgeworben wird.
        Eine andere Ursache für die Fluktuation von Mitarbeitern ist,
        dass diese an einem Projekt beteiligt sind,
        welches von dem Unternehmen nicht weiterverfolgt werden soll.
        Im Gegensatz zum Unternehmen glauben die Mitarbeiter jedoch an den Erfolg ihrer Idee
        und entschließen sich diese in einem eigenen oder einem anderen Unternehmen zu realisieren.

        Unabhängig von der Art wie der Mitarbeiter das Unternehmen verlässt,
        verliert das Unternehmen so einen Teil seines geistigen Eigentums
        und kann daraus keine Innovationen und Gewinne mehr generieren.
        Schlimmer noch, ein anderes Unternehmen profitiert von den bereits getätigten Investitionen.

    \item[Anzahl erfolgreicher Ideen]
        Nach \cite{stevens19973} werden ca. 3000 Ideen benötigt, um am Ende ein erfolgreiches Produkt zu erhalten.
        Aus diesen müssen die relevanten Ideen ausgewählt und untersucht werden.
        Angepasst auf den dreistufigen Innovationsprozess verlassen nur etwa 125 dieser Ideen die erste Stufe
        und werden in kleinen Projekten entwickelt.
        Im weiteren Verlauf werden von diesen 125 Projekten im Schnitt überschaubare 1,7 Projekte kommerzialisiert.
        Es müssen somit viele Ideen gleichzeitig entwickelt werden,
        von welchen nur ein geringer Teil später Gewinne einbringt.
        Viele der Ideen werden nicht umgesetzt, da deren eigentliches Potential nicht ersichtlich ist.
        Insgesamt entstehen somit hohe verschwendete Kosten und potentielle Gewinne werden nicht erwirtschaftet.

    \item[Ausschluss von externem Wissen]
        In der IT-Branche ist das Wissen auf viele Unternehmen und vor allem zahlreiche Startups verteilt.
        Außerdem ist viel Wissen beispielsweise durch Open-Source-Software frei zugänglich.
        Wird der Ansatz der \textit{Closed Innovation} streng verfolgt,
        so kann auf dieses Wissen jedoch nicht zugegriffen werden.
\end{description}