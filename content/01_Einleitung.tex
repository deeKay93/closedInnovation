\section{Einleitung}\label{sec:einleitung}

Innovationen sind einer der wichtigsten Aspekte für ein Unternehmen um erfolgreich zu sein und zu bleiben.
Von Beginn an ist das Gebiet der Informationstechnologie von Innovationen getrieben.

Große Meilensteine wie zum Beispiel die ersten Computer, Smartphones
und künstliche Intelligenz begannen als eine Idee und wurde schließlich zur Innovation.

In der Praxis finden sich viele verschiedene Ansätze,
wie sich für ein Unternehmen ein \textit{Innovationsprozess} etablieren lässt.

Henry W. Chesbrough führt in \cite{chesbrough2003} erstmals die Begriffe \textit{Open} und
\textit{Closed Innovation} ein.
Mit diesen Begriffen benennt er zwei Paradigmen, welche den Innovationsprozess im Hinblick auf die
Interaktion mit der Umwelt des Unternehmens und den Fluss von Wissen betrachtet.

Die \textit{Closed Innovation}, mit welcher sich diese Seminararbeit hauptsächlich befasst,
begrenzt sich hierbei auf das eigene Unternehmen, während sich die \textit{Open Innovation}
nach außen hin öffnet.