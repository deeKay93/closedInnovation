\section{Einleitung}\label{sec:einleitung}

Innovationen sind einer der wichtigsten Aspekte für ein Unternehmen um erfolgreich zu sein und zu bleiben.
Von Beginn an ist das Gebiet der Informationstechnologie von solchen Innovationen getrieben.

Vom ersten Computer über Smartphones bis hin zu Themen wie künstlicher Intelligenz,
all dies begann als eine Idee und wurde schließlich zur Innovation.

In der Praxis finden sich viele verschiedene Arten,
wie sich für ein Unternehmen ein sogenannter \textit{Innovationsprozess} etablieren lässt.

In \cite{chesbrough2003} führt Henry W. Chesbrough erstmals die Begriffe \textit{Open} und
\textit{Closed Innovation} ein.
Mit diesen Begriffen benennt er zwei Paradigmen, welche den Innovationsprozess im Hinblick auf die
Interaktion mit der Umwelt des Unternehmens und dem Fluss von Wissen betrachtet.

Die \textit{Closed Innovation}, mit welcher sich diese Seminararbeit hauptsächlich befasst,
fokusiert sich hierbei auf das eigene Unternehmen, während sich die \textit{Open Innovation}
nach außen hin öffnet.

In dem folgenden Kapitel wird die \textit{Open Innovation} sowie grundlegend deren Gegenstück betrachtet.
Doch zunächst folgt die Erläuterung der Grundbegriffe \textit{Innovation} und \textit{Innovationsprozess}.