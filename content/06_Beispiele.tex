\section{Beispiele}\label{sec:beispiele}

\subsection{Verschiedene Unternehmen in der IT-Branche}\label{sec:beispiele-unternehmen}
Chesbrough beschreibt in \cite{chesbrough2003} verschiedene Innovationsansätze von Unternehmen.
Im Folgenden werden die \textit{Closed Innovation}-Ansätze der Unternehmen \textit{Xerox} und \textit{IBM} behandelt.

\paragraph{Xerox \cite[1\psqq]{chesbrough2003}}
Als führendes Unternehmen der amerikanischen Drucker- und Kopiererbranche,
ist ein Xerox Beispiel für die Anwendung von \textit{Closed Innovation}.
1970 gründet das Unternehmen das \ac{parc}.
Dort wurden für die Informationstechnologie wegweisende Technologien wie
Ethernet, graphische Benutzeroberflächen, Textverarbeitungsprogramme und Laserdrucker geboren.

Hierbei ist auffällig, dass - mit Ausnahme der Laserdrucker - keine der Technologien mit Xerox
assoziiert werden.
Dem Unternehmen ist es nicht gelungen einen Nutzen aus diesen Innovationen zu ziehen.

Durch Übernahme von Mitarbeiter gelangt beispielsweise das Wissen zu graphischen Benutzeroberflächen bei Apple
und das Textverarbeitungsprogramm wird bei Microsoft zu dem heute weit verbreiteten \textit{Microsoft Word}.

Ethernet hingegen verlässt das Unternehmen durch dessen Erfinder Robert Metcalfe,
da er nicht warten wollte bis Xerox diese Technologie kommerzialisiert \cite[81]{chesbrough2003}.
Auf ähnliche Weiße ensteht das bekannte Unternehmen \textit{Adobe}.



\paragraph{IBM \cite[93\psqq]{chesbrough2003}}
Mit der Gründung eines Forschungszentrum im Jahre 1945 verfolgt das Technologieunternehmen ebenfalls den \textit{Closed Innovation}-Ansatz.
Zu den Innovationen IBMs zählen beispielsweise
\textit{RAMAC 305} -- die erste Festplatte,
\textit{FORTRAN} -- die erste höhere Programmiersprache und
Magnetbänder.

1964 definiert IBM durch das \textit{System 360} den damaligen Standard für Mainframe-Computer (Großrechner).
Für \textit{Closed Innovation} typisch entwickelt IBM nicht nur das System an sich,
sondern unter anderem auch Schlüsselkomponenten, das Betriebssystem, Software und sogar die Tastatur und die Stromversorgung.
IBM war zu diesen Entwicklungen, da das gesamte System auf einer komplett neuen Architektur basiert.
Um auf Zulieferer zurückzugreifen müsste diese zuvor kosten- und zeitintensiv geschult werden.

Im Vergleich zu Xerox ist IBM mit diesem Ansatz deutlich erfolgreicher.
Ab 1980 wird dieser Erfolg jedoch gebremst.
Durch die geänderte Wissenslandschaft im Bereich der Informatik haben auch andere Unternehmen
Zugriff auf notwendiges Wissen und können ihre Ideen vermarkten.
IBM vierliert so seine Alleinstellung als \enquote{Ideenmaschine}
und beginnt die Transformation von der \textit{Closed} zur \textit{Open Innovation}.


\subsection{SAP}\label{sec:beispiele-sap}
Im Jahr 1972 gründen Claus Wellenreuther, Hans-Werner Hector, Klaus Tschira, Dietmar Hopp und Hasso Plattner das Unternehmen.
Ziel der fünf ehemaligen IBM Mitarbeiter war es
Standardsoftware zur Echtzeit-\linebreak Verarbeitung von Daten für andere Unternehmen zu entwickeln.

Diese Entwicklung sollte eigentlich innerhalb von IBM geschehen.
Jedoch gestatte das Unternehmen dies nicht und wollte die Entwicklung durch andere Mitarbeiter durchführen lassen.
Daraufhin entschlossen sich die fünf SAP-Gründer ihre Idee in einem eigenen Unternehmen umzusetzen. \cite{SAPCompa72:online}

\paragraph{Gesamtes Unternehmen}\label{sec:beispiele-sap-gesamt}
Das Unternehmen SAP als solches pflegt eine überwiegend offene Unternehmenskultur.
Es ist bekannt dafür durch zahlreiche Akquisitionen das unternehmensinterne geistige Eigentum zu erweitern.

Durch die Veröffentlichung von Open-Source-Software sind zahlreiche Bibliotheken wie \textit{OpenUI5} offen verfügbar.

Auch \enquote{klassische} SAP-Anwendungen sind auf Offentheit ausgelegt.
Durch entsprechende Schnittstellen und Tools wird es sogenannten \enquote{Partnern}
ermöglicht das vorhandene Programm zu erweitern.

Das Unternehmen pflegt zudem eine intensive Kooperation mit anderen Unternehmen
wie Apple \cite{Appleund81:online} und Microsoft \cite{Microsof58:online}.
Kunden werden mit Hilfe von \textit{Design Thinking} \cite{SAPDesig64:online} früh in den Entwicklungsprozess eingebunden.



\paragraph{\ac{sac}}\label{sec:beispiele-sap-sac}
Hinter diesen Begriff steckt eine Analyse- und Planungssoftware des Unternehmens SAP.
Als \enquote{Cloudsoftware} wird die Anwendung auf SAP-eigenen Servern zur Verfügung gestellt.
Auch die regelmäßige Aktualisierung der Software gehört zu den Aufgaben der SAP.

Im Fall von \ac{sac} wird dies alle zwei Wochen durchgeführt.
Diese kurzen Entwicklungs- und damit Feedback-Zyklen führen zu einem regen Austausch zwischen der Entwicklungsabteilung und den Kunden.
Mit ausgewählten Kunden wird diese Kooperation noch weiter intensiviert.

Ein weiterer Indikator für die Offentheit der Entwicklung sind die verwendeten Technologien.
Grundlegend basiert die Anwendung zwar auf SAP-eigenen Technologien,
aber es werden auch am Markt etablierte Bibliotheken wie \textit{React} und andere quelloffene Bausteine verwendet.
